\documentclass[ignorenonframetext,]{beamer}
\setbeamertemplate{caption}[numbered]
\setbeamertemplate{caption label separator}{: }
\setbeamercolor{caption name}{fg=normal text.fg}
\beamertemplatenavigationsymbolsempty
\usepackage{lmodern}
\usepackage{amssymb,amsmath}
\usepackage{ifxetex,ifluatex}
\usepackage{fixltx2e} % provides \textsubscript
\ifnum 0\ifxetex 1\fi\ifluatex 1\fi=0 % if pdftex
  \usepackage[T1]{fontenc}
  \usepackage[utf8]{inputenc}
\else % if luatex or xelatex
  \ifxetex
    \usepackage{mathspec}
  \else
    \usepackage{fontspec}
  \fi
  \defaultfontfeatures{Ligatures=TeX,Scale=MatchLowercase}
\fi
\usefonttheme{serif}
% use upquote if available, for straight quotes in verbatim environments
\IfFileExists{upquote.sty}{\usepackage{upquote}}{}
% use microtype if available
\IfFileExists{microtype.sty}{%
\usepackage{microtype}
\UseMicrotypeSet[protrusion]{basicmath} % disable protrusion for tt fonts
}{}
\newif\ifbibliography
\hypersetup{
            pdftitle={Gráficos Básicos en R},
            pdfauthor={Paula Reyes},
            pdfborder={0 0 0},
            breaklinks=true}
\urlstyle{same}  % don't use monospace font for urls
\usepackage{color}
\usepackage{fancyvrb}
\newcommand{\VerbBar}{|}
\newcommand{\VERB}{\Verb[commandchars=\\\{\}]}
\DefineVerbatimEnvironment{Highlighting}{Verbatim}{commandchars=\\\{\}}
% Add ',fontsize=\small' for more characters per line
\usepackage{framed}
\definecolor{shadecolor}{RGB}{248,248,248}
\newenvironment{Shaded}{\begin{snugshade}}{\end{snugshade}}
\newcommand{\KeywordTok}[1]{\textcolor[rgb]{0.13,0.29,0.53}{\textbf{#1}}}
\newcommand{\DataTypeTok}[1]{\textcolor[rgb]{0.13,0.29,0.53}{#1}}
\newcommand{\DecValTok}[1]{\textcolor[rgb]{0.00,0.00,0.81}{#1}}
\newcommand{\BaseNTok}[1]{\textcolor[rgb]{0.00,0.00,0.81}{#1}}
\newcommand{\FloatTok}[1]{\textcolor[rgb]{0.00,0.00,0.81}{#1}}
\newcommand{\ConstantTok}[1]{\textcolor[rgb]{0.00,0.00,0.00}{#1}}
\newcommand{\CharTok}[1]{\textcolor[rgb]{0.31,0.60,0.02}{#1}}
\newcommand{\SpecialCharTok}[1]{\textcolor[rgb]{0.00,0.00,0.00}{#1}}
\newcommand{\StringTok}[1]{\textcolor[rgb]{0.31,0.60,0.02}{#1}}
\newcommand{\VerbatimStringTok}[1]{\textcolor[rgb]{0.31,0.60,0.02}{#1}}
\newcommand{\SpecialStringTok}[1]{\textcolor[rgb]{0.31,0.60,0.02}{#1}}
\newcommand{\ImportTok}[1]{#1}
\newcommand{\CommentTok}[1]{\textcolor[rgb]{0.56,0.35,0.01}{\textit{#1}}}
\newcommand{\DocumentationTok}[1]{\textcolor[rgb]{0.56,0.35,0.01}{\textbf{\textit{#1}}}}
\newcommand{\AnnotationTok}[1]{\textcolor[rgb]{0.56,0.35,0.01}{\textbf{\textit{#1}}}}
\newcommand{\CommentVarTok}[1]{\textcolor[rgb]{0.56,0.35,0.01}{\textbf{\textit{#1}}}}
\newcommand{\OtherTok}[1]{\textcolor[rgb]{0.56,0.35,0.01}{#1}}
\newcommand{\FunctionTok}[1]{\textcolor[rgb]{0.00,0.00,0.00}{#1}}
\newcommand{\VariableTok}[1]{\textcolor[rgb]{0.00,0.00,0.00}{#1}}
\newcommand{\ControlFlowTok}[1]{\textcolor[rgb]{0.13,0.29,0.53}{\textbf{#1}}}
\newcommand{\OperatorTok}[1]{\textcolor[rgb]{0.81,0.36,0.00}{\textbf{#1}}}
\newcommand{\BuiltInTok}[1]{#1}
\newcommand{\ExtensionTok}[1]{#1}
\newcommand{\PreprocessorTok}[1]{\textcolor[rgb]{0.56,0.35,0.01}{\textit{#1}}}
\newcommand{\AttributeTok}[1]{\textcolor[rgb]{0.77,0.63,0.00}{#1}}
\newcommand{\RegionMarkerTok}[1]{#1}
\newcommand{\InformationTok}[1]{\textcolor[rgb]{0.56,0.35,0.01}{\textbf{\textit{#1}}}}
\newcommand{\WarningTok}[1]{\textcolor[rgb]{0.56,0.35,0.01}{\textbf{\textit{#1}}}}
\newcommand{\AlertTok}[1]{\textcolor[rgb]{0.94,0.16,0.16}{#1}}
\newcommand{\ErrorTok}[1]{\textcolor[rgb]{0.64,0.00,0.00}{\textbf{#1}}}
\newcommand{\NormalTok}[1]{#1}
\usepackage{graphicx,grffile}
\makeatletter
\def\maxwidth{\ifdim\Gin@nat@width>\linewidth\linewidth\else\Gin@nat@width\fi}
\def\maxheight{\ifdim\Gin@nat@height>\textheight0.8\textheight\else\Gin@nat@height\fi}
\makeatother
% Scale images if necessary, so that they will not overflow the page
% margins by default, and it is still possible to overwrite the defaults
% using explicit options in \includegraphics[width, height, ...]{}
\setkeys{Gin}{width=\maxwidth,height=\maxheight,keepaspectratio}

% Prevent slide breaks in the middle of a paragraph:
\widowpenalties 1 10000
\raggedbottom

\AtBeginPart{
  \let\insertpartnumber\relax
  \let\partname\relax
  \frame{\partpage}
}
\AtBeginSection{
  \ifbibliography
  \else
    \let\insertsectionnumber\relax
    \let\sectionname\relax
    \frame{\sectionpage}
  \fi
}
\AtBeginSubsection{
  \let\insertsubsectionnumber\relax
  \let\subsectionname\relax
  \frame{\subsectionpage}
}

\setlength{\parindent}{0pt}
\setlength{\parskip}{6pt plus 2pt minus 1pt}
\setlength{\emergencystretch}{3em}  % prevent overfull lines
\providecommand{\tightlist}{%
  \setlength{\itemsep}{0pt}\setlength{\parskip}{0pt}}
\setcounter{secnumdepth}{0}


\usepackage{graphicx}
\usepackage{rotating}
%\setbeamertemplate{caption}[numbered]
\usepackage{hyperref}
\usepackage{caption}
\usepackage{tikz}
\usepackage[normalem]{ulem}
%\mode<presentation>
\usepackage{wasysym}
\usepackage{amsmath}

\setbeamertemplate{navigation symbols}{}
\institute{Agrosavia}
\titlegraphic{\includegraphics[width=0.3\paperwidth]{\string./images/Imagen2.png}}
%\setbeamertemplate{title page}[empty]






\setbeamertemplate{background}
{\includegraphics[width=\paperwidth,height=\paperheight,keepaspectratio]{\string./images/Imagen2_White_Background.png}}


\pgfdeclareimage[width=\paperwidth]{mybackground}{\string./images/Imagen1_White_Background.png}

\setbeamertemplate{title page}{

        \begin{picture}(0,0)

            \put(-28.9,-157.5){%
                \pgfuseimage{mybackground}
            }

            \put(-165,-130){%
                \begin{minipage}[b][45mm][t]{226mm}
\begin{center}
                    \usebeamerfont{title}{\inserttitle\par}
\usebeamerfont{subtitle}{\insertsubtitle\par}
\vspace{1cm}
\usebeamerfont{author}{\insertauthor\par}
\usebeamerfont{institute}{\insertinstitute\par}
\vspace{1cm}
\usebeamerfont{date}{\insertdate\par}
\end{center}
                \end{minipage}
            }

            \end{picture}

    }



\setbeamerfont{subtitle}{size=\small}
\setbeamerfont{title}{size=\large}

\setbeamerfont{frametitle}{size=\large}

\setbeamerfont{normal text}{size=\small}

\setbeamercovered{transparent}

\definecolor{agrosaviablue}{HTML}{004F9F}
\definecolor{agrosaviagreen}{HTML}{11A344}

\setbeamercolor{frametitle}{fg=agrosaviablue}
\setbeamercolor{title}{fg=agrosaviablue}
\setbeamercolor{local structure}{fg=agrosaviablue}
\setbeamercolor{section in toc}{fg=agrosaviablue,bg=white}
\setbeamercolor{subsection in toc}{fg=agrosaviablue,bg=white}
\setbeamercolor{item projected}{fg=agrosaviagreen,bg=white}
\setbeamertemplate{itemize item}{\color{agrosaviagreen}$\bullet$}
\setbeamertemplate{itemize subitem}{\color{agrosaviagreen}\scriptsize{$\bullet$}}
\setbeamercolor{local structure}{fg=agrosaviagreen}

%\let\Tiny=\tiny

%\let\Tiny\tiny

\AtBeginPart{}
\AtBeginSection{}
\AtBeginSubsection{}
\AtBeginSubsubsection{}
\setlength{\emergencystretch}{0em}
\setlength{\parskip}{0pt}


\AtBeginEnvironment{Shaded}{\scriptsize}% Step font down one size relative to current font
\AtBeginEnvironment{verbatim}{\scriptsize}% Step font down one size relative to current font

\AtBeginDocument{\usebeamerfont{normal text}} % Use normal text font previously defined

\title{Gráficos Básicos en R}
\subtitle{Módulo 1 - Exploración de datos mediante gráficas}
\author{Paula Reyes}
\institute{AGROSAVIA}
\date{}

\begin{document}
\frame{\titlepage}

\begin{frame}
\tableofcontents[hideallsubsections]
\end{frame}

\section{Diseño de gráfica - propuesta
estudiantes}\label{diseno-de-grafica---propuesta-estudiantes}

\begin{frame}{Diseño de gráfica - propuesta estudiantes}

¿Qué hacen ustedes cuando tienen varias muestras de una misma variable y
quieren ver la distibución?

\begin{enumerate}
\def\labelenumi{(\alph{enumi})}
\tightlist
\item
  Excel
\item
  SAS (licencia)
\end{enumerate}

\end{frame}

\begin{frame}{Diseño de gráfica - propuesta estudiantes}

Usaremos un conjunto de datos de R \textit{PlantGrowth} resultados de un
experimento de pesos de plantas en tres condiciones. Inicialmente
queremos ver habilidades iniciales de los participantes del curso.

Conjunto de datos
\underline{\textcolor{blue}{\href{https://corpoicaorg.sharepoint.com/:x:/s/academiaagrosaviacurso.r/EfMk4erbqc9NmgwUukiYUIUBCy2gw9arMpAQbzHH3Cqq9Q?e=mJwDfK} {PlantGrowth}}}

De este conjunto de datos obtener gráficas de distribución de datos.

Enviar figuras al correo
\href{mailto:academia@agrosavia.co}{\nolinkurl{academia@agrosavia.co}}.
Figura primer día - Nombre.

\end{frame}

\section{Tipos de diagramas y correspondencia con análisis
estadísticos}\label{tipos-de-diagramas-y-correspondencia-con-analisis-estadisticos}

\begin{frame}{Tipos de diagramas - ¿Cómo seleccionar?}

Cómo escogemos cada uno el tipo de gráfica a usar? Para poder convertir
datos en conocimiento es importante poder presentar de la manera
correcta (clara y concisa)

\includegraphics[height=5cm]{images/selectchart.png}

\underline{\tiny{\textcolor{blue}{\href{http://bigdata.black/analytics-predictions/visual-analytics/how-to-choose-the-right-chart/}{BigData is the New Black}}}}

\end{frame}

\section{Diagramas de distribución}\label{diagramas-de-distribucion}

\begin{frame}{Histograma}

En un histograma los datos estan agrupados en rangos (por ejemplo
0-0.5,0.5-1). Se grafican los rangos como barras continuas, cada barra
representa un rango.

\begin{itemize}
\item El ancho de la barra es proporcional al ancho del rango. 
\item La altura de la barra es proporcional a la cantidad de datos en ese rango (frecuencia)
\end{itemize}

\includegraphics{Agrosavia_Theme_Graficos_Curso2_files/figure-beamer/unnamed-chunk-1-1.pdf}
Es una representación gráfica de la distribución de los datos. Permite
observar tenedencias (eg. donde estan agrupados la mayor parte de los
datos).

\end{frame}

\begin{frame}[fragile]{El histograma de frecuencias - datos - estructura
- \textit{PlantGrowth}}

Usaremos un conjunto de datos \textit{PlantGrowth} resultado de un
experimento de pesos de plantas en tres condiciones (control - ctrl,
tratamiento1 -trt1, tratamiento2 -trt2)

\begin{Shaded}
\begin{Highlighting}[]
\KeywordTok{str}\NormalTok{(PlantGrowth)}
\end{Highlighting}
\end{Shaded}

\begin{verbatim}
## 'data.frame':    30 obs. of  2 variables:
##  $ weight: num  4.17 5.58 5.18 6.11 4.5 4.61 5.17 4.53 5.33 5.14 ...
##  $ group : Factor w/ 3 levels "ctrl","trt1",..: 1 1 1 1 1 1 1 1 1 1 ...
\end{verbatim}

\end{frame}

\begin{frame}[fragile]{El histograma de frecuencias - datos}

Usaremos un conjunto de datos \textit{PlantGrowth} resultados de un
experimento de pesos de plantas en tres condiciones (control - ctrl,
tratamiento1 -trt1, tratamiento2 -trt2)

\begin{Shaded}
\begin{Highlighting}[]
\CommentTok{#PlantGrowth}
\KeywordTok{head}\NormalTok{(PlantGrowth)}
\end{Highlighting}
\end{Shaded}

\begin{verbatim}
##   weight group
## 1   4.17  ctrl
## 2   5.58  ctrl
## 3   5.18  ctrl
## 4   6.11  ctrl
## 5   4.50  ctrl
## 6   4.61  ctrl
\end{verbatim}

\end{frame}

\begin{frame}[fragile]{El histograma de frecuencias}

\begin{Shaded}
\begin{Highlighting}[]
\KeywordTok{hist}\NormalTok{(PlantGrowth}\OperatorTok{$}\NormalTok{weight)}
\end{Highlighting}
\end{Shaded}

\includegraphics{Agrosavia_Theme_Graficos_Curso2_files/figure-beamer/unnamed-chunk-4-1.pdf}

\end{frame}

\begin{frame}[fragile]{El histograma de frecuencias - estructura}

\begin{Shaded}
\begin{Highlighting}[]
\KeywordTok{str}\NormalTok{(}\KeywordTok{hist}\NormalTok{(PlantGrowth}\OperatorTok{$}\NormalTok{weight,}\DataTypeTok{plot=}\OtherTok{FALSE}\NormalTok{))}
\end{Highlighting}
\end{Shaded}

\begin{verbatim}
## List of 6
##  $ breaks  : num [1:7] 3.5 4 4.5 5 5.5 6 6.5
##  $ counts  : int [1:6] 2 5 6 9 4 4
##  $ density : num [1:6] 0.133 0.333 0.4 0.6 0.267 ...
##  $ mids    : num [1:6] 3.75 4.25 4.75 5.25 5.75 6.25
##  $ xname   : chr "PlantGrowth$weight"
##  $ equidist: logi TRUE
##  - attr(*, "class")= chr "histogram"
\end{verbatim}

\end{frame}

\begin{frame}[fragile]{El histograma de frecuencias - número de rangos}

\begin{Shaded}
\begin{Highlighting}[]
\KeywordTok{hist}\NormalTok{(PlantGrowth}\OperatorTok{$}\NormalTok{weight,}\DataTypeTok{breaks=}\DecValTok{20}\NormalTok{)}
\end{Highlighting}
\end{Shaded}

\includegraphics{Agrosavia_Theme_Graficos_Curso2_files/figure-beamer/unnamed-chunk-6-1.pdf}

\end{frame}

\begin{frame}[fragile]{El histograma de frecuencias - número de rangos}

\begin{Shaded}
\begin{Highlighting}[]
\KeywordTok{hist}\NormalTok{(PlantGrowth}\OperatorTok{$}\NormalTok{weight,}\DataTypeTok{breaks=}\KeywordTok{c}\NormalTok{(}\DecValTok{1}\OperatorTok{:}\DecValTok{7}\NormalTok{))}
\end{Highlighting}
\end{Shaded}

\includegraphics{Agrosavia_Theme_Graficos_Curso2_files/figure-beamer/unnamed-chunk-7-1.pdf}

\end{frame}

\begin{frame}[fragile]{El histograma de frecuencias - valores por cada
rango}

\begin{Shaded}
\begin{Highlighting}[]
\KeywordTok{hist}\NormalTok{(PlantGrowth}\OperatorTok{$}\NormalTok{weight,}\DataTypeTok{col=}\KeywordTok{c}\NormalTok{(}\StringTok{"gold"}\NormalTok{),}\DataTypeTok{labels=}\OtherTok{TRUE}\NormalTok{)}
\end{Highlighting}
\end{Shaded}

\includegraphics{Agrosavia_Theme_Graficos_Curso2_files/figure-beamer/unnamed-chunk-8-1.pdf}

\end{frame}

\begin{frame}[fragile]{El histograma de frecuencias - color}

\begin{Shaded}
\begin{Highlighting}[]
\KeywordTok{hist}\NormalTok{(PlantGrowth}\OperatorTok{$}\NormalTok{weight,}\DataTypeTok{col=}\KeywordTok{c}\NormalTok{(}\StringTok{"gold"}\NormalTok{))}
\end{Highlighting}
\end{Shaded}

\includegraphics{Agrosavia_Theme_Graficos_Curso2_files/figure-beamer/unnamed-chunk-9-1.pdf}
\underline{\tiny{\textcolor{blue}{\href{http://www.stat.columbia.edu/~tzheng/files/Rcolor.pdf}{Rcolor}}}}

\end{frame}

\begin{frame}[fragile]{El histograma de frecuencias - color borde}

\begin{Shaded}
\begin{Highlighting}[]
\KeywordTok{hist}\NormalTok{(PlantGrowth}\OperatorTok{$}\NormalTok{weight,}\DataTypeTok{col=}\KeywordTok{c}\NormalTok{(}\StringTok{"gold"}\NormalTok{),}\DataTypeTok{border=}\StringTok{"orange"}\NormalTok{)}
\end{Highlighting}
\end{Shaded}

\includegraphics{Agrosavia_Theme_Graficos_Curso2_files/figure-beamer/unnamed-chunk-10-1.pdf}
\underline{\tiny{\textcolor{blue}{\href{http://www.stat.columbia.edu/~tzheng/files/Rcolor.pdf}{Rcolor}}}}

\end{frame}

\begin{frame}[fragile]{El histograma de frecuencias - cambiando ejes}

\begin{Shaded}
\begin{Highlighting}[]
\KeywordTok{hist}\NormalTok{(PlantGrowth}\OperatorTok{$}\NormalTok{weight,}\DataTypeTok{col=}\KeywordTok{c}\NormalTok{(}\StringTok{"gold"}\NormalTok{),}\DataTypeTok{xlim=}\KeywordTok{c}\NormalTok{(}\DecValTok{0}\NormalTok{,}\DecValTok{10}\NormalTok{),}\DataTypeTok{ylim=}\KeywordTok{c}\NormalTok{(}\DecValTok{0}\NormalTok{,}\DecValTok{10}\NormalTok{))}
\end{Highlighting}
\end{Shaded}

\includegraphics{Agrosavia_Theme_Graficos_Curso2_files/figure-beamer/unnamed-chunk-11-1.pdf}

\end{frame}

\begin{frame}[fragile]{El histograma de frecuencias - título (gráfico y
eje x)}

\begin{Shaded}
\begin{Highlighting}[]
\KeywordTok{hist}\NormalTok{(PlantGrowth}\OperatorTok{$}\NormalTok{weight,}\DataTypeTok{col=}\KeywordTok{c}\NormalTok{(}\StringTok{"gold"}\NormalTok{),}\DataTypeTok{main =}\StringTok{"Plant Growth"}\NormalTok{,}\DataTypeTok{xlab=}\StringTok{"Weight (kg)"}\NormalTok{)}
\KeywordTok{grid}\NormalTok{(}\DecValTok{5}\NormalTok{)}
\end{Highlighting}
\end{Shaded}

\includegraphics{Agrosavia_Theme_Graficos_Curso2_files/figure-beamer/unnamed-chunk-12-1.pdf}

\end{frame}

\begin{frame}[fragile]{El histograma de frecuencias - con estima de
densidad}

\begin{Shaded}
\begin{Highlighting}[]
\KeywordTok{hist}\NormalTok{(PlantGrowth}\OperatorTok{$}\NormalTok{weight,}\DataTypeTok{col=}\KeywordTok{c}\NormalTok{(}\StringTok{"gold"}\NormalTok{),}\DataTypeTok{prob=}\OtherTok{TRUE}\NormalTok{)}
\KeywordTok{lines}\NormalTok{(}\KeywordTok{density}\NormalTok{(PlantGrowth}\OperatorTok{$}\NormalTok{weight))}
\KeywordTok{lines}\NormalTok{(}\KeywordTok{density}\NormalTok{(PlantGrowth}\OperatorTok{$}\NormalTok{weight,}\DataTypeTok{adjust =} \DecValTok{2}\NormalTok{),}\DataTypeTok{lty=}\StringTok{"dotted"}\NormalTok{)}
\KeywordTok{grid}\NormalTok{(}\DecValTok{5}\NormalTok{)}
\end{Highlighting}
\end{Shaded}

\includegraphics{Agrosavia_Theme_Graficos_Curso2_files/figure-beamer/unnamed-chunk-13-1.pdf}

\end{frame}

\begin{frame}[fragile]{Histogramas por tratamiento}

\begin{Shaded}
\begin{Highlighting}[]
\KeywordTok{levels}\NormalTok{(PlantGrowth}\OperatorTok{$}\NormalTok{group)}
\end{Highlighting}
\end{Shaded}

\begin{verbatim}
## [1] "ctrl" "trt1" "trt2"
\end{verbatim}

\begin{Shaded}
\begin{Highlighting}[]
\KeywordTok{levels}\NormalTok{(PlantGrowth}\OperatorTok{$}\NormalTok{group)[}\DecValTok{1}\NormalTok{]}
\end{Highlighting}
\end{Shaded}

\begin{verbatim}
## [1] "ctrl"
\end{verbatim}

\begin{Shaded}
\begin{Highlighting}[]
\NormalTok{PlantGrowthCtrl<-PlantGrowth[PlantGrowth}\OperatorTok{$}\NormalTok{group}\OperatorTok{==}\KeywordTok{levels}\NormalTok{(PlantGrowth}\OperatorTok{$}\NormalTok{group)[}\DecValTok{1}\NormalTok{],]}
\NormalTok{PlantGrowthTrt1<-PlantGrowth[PlantGrowth}\OperatorTok{$}\NormalTok{group}\OperatorTok{==}\KeywordTok{levels}\NormalTok{(PlantGrowth}\OperatorTok{$}\NormalTok{group)[}\DecValTok{2}\NormalTok{],]}
\NormalTok{PlantGrowthTrt2<-PlantGrowth[PlantGrowth}\OperatorTok{$}\NormalTok{group}\OperatorTok{==}\KeywordTok{levels}\NormalTok{(PlantGrowth}\OperatorTok{$}\NormalTok{group)[}\DecValTok{3}\NormalTok{],]}
\end{Highlighting}
\end{Shaded}

\end{frame}

\begin{frame}[fragile]{Histogramas por tratamiento}

\begin{Shaded}
\begin{Highlighting}[]
\KeywordTok{par}\NormalTok{(}\KeywordTok{c}\NormalTok{(}\DecValTok{3}\NormalTok{,}\DecValTok{1}\NormalTok{))}
\KeywordTok{hist}\NormalTok{(PlantGrowthCtrl}\OperatorTok{$}\NormalTok{weight, }\DataTypeTok{xlim=}\KeywordTok{c}\NormalTok{(}\DecValTok{3}\NormalTok{,}\DecValTok{7}\NormalTok{), }\DataTypeTok{col=}\KeywordTok{rgb}\NormalTok{(}\DecValTok{1}\NormalTok{,}\DecValTok{0}\NormalTok{,}\DecValTok{0}\NormalTok{,}\FloatTok{0.4}\NormalTok{), }
     \DataTypeTok{xlab=}\StringTok{"weight"}\NormalTok{, }\DataTypeTok{main=}\StringTok{"Plant Growth distribution Control"}\NormalTok{)}
\KeywordTok{hist}\NormalTok{(PlantGrowthTrt1}\OperatorTok{$}\NormalTok{weight, }\DataTypeTok{xlim=}\KeywordTok{c}\NormalTok{(}\DecValTok{3}\NormalTok{,}\DecValTok{7}\NormalTok{), }\DataTypeTok{col=}\KeywordTok{rgb}\NormalTok{(}\DecValTok{0}\NormalTok{,}\DecValTok{0}\NormalTok{,}\DecValTok{1}\NormalTok{,}\FloatTok{0.4}\NormalTok{),}
  \DataTypeTok{xlab=}\StringTok{"weight"}\NormalTok{, }\DataTypeTok{main=}\StringTok{"Plant Growth distribution Treatment 1"}\NormalTok{)}
\KeywordTok{hist}\NormalTok{(PlantGrowthTrt2}\OperatorTok{$}\NormalTok{weight, }\DataTypeTok{xlim=}\KeywordTok{c}\NormalTok{(}\DecValTok{3}\NormalTok{,}\DecValTok{7}\NormalTok{), }\DataTypeTok{col=}\KeywordTok{rgb}\NormalTok{(}\DecValTok{0}\NormalTok{,}\DecValTok{1}\NormalTok{,}\DecValTok{0}\NormalTok{,}\FloatTok{0.4}\NormalTok{),}
  \DataTypeTok{xlab=}\StringTok{"weight"}\NormalTok{, }\DataTypeTok{main=}\StringTok{"Plant Growth distribution Treatment 2"}\NormalTok{)   }
\end{Highlighting}
\end{Shaded}

\end{frame}

\begin{frame}{Histogramas por tratamiento}

\includegraphics[height=10cm]{images/Par_PlantGrowth_byTreatment.png}

\end{frame}

\begin{frame}[fragile]{Histogramas por tratamiento}

\begin{Shaded}
\begin{Highlighting}[]
\KeywordTok{hist}\NormalTok{(PlantGrowthCtrl}\OperatorTok{$}\NormalTok{weight, }\DataTypeTok{xlim=}\KeywordTok{c}\NormalTok{(}\DecValTok{3}\NormalTok{,}\DecValTok{7}\NormalTok{), }\DataTypeTok{col=}\KeywordTok{rgb}\NormalTok{(}\DecValTok{1}\NormalTok{,}\DecValTok{0}\NormalTok{,}\DecValTok{0}\NormalTok{,}\FloatTok{0.5}\NormalTok{), }
     \DataTypeTok{xlab=}\StringTok{"weight"}\NormalTok{,}\DataTypeTok{ylab=}\StringTok{"nbr of plants"}\NormalTok{, }\DataTypeTok{main=}\StringTok{"Plant Growth distribution"}\NormalTok{ )}
\KeywordTok{hist}\NormalTok{(PlantGrowthTrt1}\OperatorTok{$}\NormalTok{weight, }\DataTypeTok{xlim=}\KeywordTok{c}\NormalTok{(}\DecValTok{3}\NormalTok{,}\DecValTok{7}\NormalTok{), }\DataTypeTok{col=}\KeywordTok{rgb}\NormalTok{(}\DecValTok{0}\NormalTok{,}\DecValTok{0}\NormalTok{,}\DecValTok{1}\NormalTok{,}\FloatTok{0.5}\NormalTok{), }\DataTypeTok{add=}\NormalTok{T)}
\KeywordTok{hist}\NormalTok{(PlantGrowthTrt2}\OperatorTok{$}\NormalTok{weight, }\DataTypeTok{xlim=}\KeywordTok{c}\NormalTok{(}\DecValTok{3}\NormalTok{,}\DecValTok{7}\NormalTok{), }\DataTypeTok{col=}\KeywordTok{rgb}\NormalTok{(}\DecValTok{0}\NormalTok{,}\DecValTok{1}\NormalTok{,}\DecValTok{0}\NormalTok{,}\FloatTok{0.5}\NormalTok{), }\DataTypeTok{add=}\NormalTok{T)}
\KeywordTok{legend}\NormalTok{(}\StringTok{"topright"}\NormalTok{, }\DataTypeTok{legend=}\KeywordTok{c}\NormalTok{(}\StringTok{"Control"}\NormalTok{,}\StringTok{"Treatment 1"}\NormalTok{,}\StringTok{"Treatment 2"}\NormalTok{), }
       \DataTypeTok{col=}\KeywordTok{c}\NormalTok{(}\KeywordTok{rgb}\NormalTok{(}\DecValTok{1}\NormalTok{,}\DecValTok{0}\NormalTok{,}\DecValTok{0}\NormalTok{,}\FloatTok{0.5}\NormalTok{),}\KeywordTok{rgb}\NormalTok{(}\DecValTok{0}\NormalTok{,}\DecValTok{0}\NormalTok{,}\DecValTok{1}\NormalTok{,}\FloatTok{0.5}\NormalTok{),}\KeywordTok{rgb}\NormalTok{(}\DecValTok{0}\NormalTok{,}\DecValTok{1}\NormalTok{,}\DecValTok{0}\NormalTok{,}\FloatTok{0.5}\NormalTok{)), }\DataTypeTok{pt.cex=}\DecValTok{2}\NormalTok{, }\DataTypeTok{pch=}\DecValTok{15}\NormalTok{ )}
\end{Highlighting}
\end{Shaded}

\end{frame}

\begin{frame}{Histogramas por tratamiento}

\includegraphics{Agrosavia_Theme_Graficos_Curso2_files/figure-beamer/unnamed-chunk-17-1.pdf}

\end{frame}

\begin{frame}{Histograma - Otras librerias}

A continuación hay ejemplos de otras librerias. Sin embargo, es
necesario instar y cargar los paquetes.

\begin{itemize}
\item lattice
\item ggplot2
\item plotly
\end{itemize}

\end{frame}

\begin{frame}[fragile]{Histograma - libreria lattice}

\begin{Shaded}
\begin{Highlighting}[]
\CommentTok{# instala libreria}
\KeywordTok{install.packages}\NormalTok{(}\StringTok{"lattice"}\NormalTok{)}
\CommentTok{# carga libreria}
\KeywordTok{library}\NormalTok{(lattice)}
\KeywordTok{histogram}\NormalTok{(}\OperatorTok{~}\NormalTok{weight,}\DataTypeTok{data=}\NormalTok{PlantGrowth)}
\end{Highlighting}
\end{Shaded}

\includegraphics[height=5cm]{images/Histogram_lattice.png}

\end{frame}

\begin{frame}[fragile]{Histograma - libreria lattice divido por grupo}

\begin{Shaded}
\begin{Highlighting}[]
\CommentTok{# instala libreria}
\KeywordTok{install.packages}\NormalTok{(}\StringTok{"lattice"}\NormalTok{)}
\CommentTok{# carga libreria}
\KeywordTok{library}\NormalTok{(lattice)}
\KeywordTok{histogram}\NormalTok{(}\OperatorTok{~}\NormalTok{weight}\OperatorTok{|}\NormalTok{group,}\DataTypeTok{data=}\NormalTok{PlantGrowth,}\DataTypeTok{type=}\StringTok{"count"}\NormalTok{,}\DataTypeTok{xlab=}\StringTok{"weight (kg)"}\NormalTok{, }
          \DataTypeTok{main=}\StringTok{"Plant Growth by treatment"}\NormalTok{,}\DataTypeTok{breaks =} \DecValTok{3}\OperatorTok{:}\DecValTok{8}\NormalTok{,}\DataTypeTok{layout=}\KeywordTok{c}\NormalTok{(}\DecValTok{3}\NormalTok{,}\DecValTok{1}\NormalTok{))}
\end{Highlighting}
\end{Shaded}

\includegraphics[height=5cm]{images/lattice_histogram31.png}

\end{frame}

\begin{frame}[fragile]{Histograma - libreria ggplot2 - qplot()}

\begin{Shaded}
\begin{Highlighting}[]
\CommentTok{# instala libreria}
\KeywordTok{install.packages}\NormalTok{(}\StringTok{"ggplot2"}\NormalTok{)}
\CommentTok{# carga libreria}
\KeywordTok{library}\NormalTok{(ggplot2)}
\KeywordTok{qplot}\NormalTok{(PlantGrowth}\OperatorTok{$}\NormalTok{weight,}\DataTypeTok{geom=}\StringTok{"histogram"}\NormalTok{,}\DataTypeTok{bins=}\DecValTok{7}\NormalTok{,}\DataTypeTok{main=}\StringTok{"Histogram for weight"}\NormalTok{,}
      \DataTypeTok{xlab=}\StringTok{"Weight"}\NormalTok{,}\DataTypeTok{fill=}\KeywordTok{I}\NormalTok{(}\StringTok{"gold"}\NormalTok{),}\DataTypeTok{col=}\KeywordTok{I}\NormalTok{(}\StringTok{"gray"}\NormalTok{),}\DataTypeTok{alpha=}\KeywordTok{I}\NormalTok{(.}\DecValTok{4}\NormalTok{),}\DataTypeTok{xlim=}\KeywordTok{c}\NormalTok{(}\DecValTok{4}\NormalTok{,}\DecValTok{7}\NormalTok{))}
\end{Highlighting}
\end{Shaded}

\includegraphics[height=5cm]{images/ggplotv1.png}

\end{frame}

\begin{frame}[fragile]{Histograma - libreria ggplot2 - ggplot()}

\begin{Shaded}
\begin{Highlighting}[]
\CommentTok{# instala libreria}
\KeywordTok{install.packages}\NormalTok{(}\StringTok{"ggplot2"}\NormalTok{)}
\CommentTok{# carga libreria}
\KeywordTok{library}\NormalTok{(ggplot2)}
\KeywordTok{ggplot}\NormalTok{(}\DataTypeTok{data=}\NormalTok{PlantGrowth,}\KeywordTok{aes}\NormalTok{(PlantGrowth}\OperatorTok{$}\NormalTok{weight))}\OperatorTok{+}
\StringTok{  }\KeywordTok{geom_histogram}\NormalTok{(}\DataTypeTok{bins =} \DecValTok{7}\NormalTok{,}\DataTypeTok{col=}\StringTok{"gray"}\NormalTok{,}\DataTypeTok{fill=}\StringTok{"gold"}\NormalTok{,}\DataTypeTok{alpha=}\FloatTok{0.4}\NormalTok{)}\OperatorTok{+}
\StringTok{  }\KeywordTok{labs}\NormalTok{(}\DataTypeTok{title =} \StringTok{"Histogram for weight"}\NormalTok{,}\DataTypeTok{x=}\StringTok{"weight"}\NormalTok{,}\DataTypeTok{y=}\StringTok{"count"}\NormalTok{,}\KeywordTok{xlim}\NormalTok{(}\KeywordTok{c}\NormalTok{(}\DecValTok{4}\NormalTok{,}\DecValTok{7}\NormalTok{)))}
\end{Highlighting}
\end{Shaded}

\includegraphics[height=5cm]{images/ggplot_histogramv2.png}

\end{frame}

\begin{frame}[fragile]{Histograma - libreria ggplot2 - forma 2 - rango
color}

\begin{Shaded}
\begin{Highlighting}[]
\KeywordTok{ggplot}\NormalTok{(}\DataTypeTok{data=}\NormalTok{PlantGrowth,}\KeywordTok{aes}\NormalTok{(PlantGrowth}\OperatorTok{$}\NormalTok{weight))}\OperatorTok{+}
\KeywordTok{geom_histogram}\NormalTok{(}\DataTypeTok{bins =} \DecValTok{7}\NormalTok{,}\DataTypeTok{col=}\StringTok{"gray"}\NormalTok{,}\KeywordTok{aes}\NormalTok{(}\DataTypeTok{fill=}\NormalTok{..count..))}\OperatorTok{+}
\KeywordTok{labs}\NormalTok{(}\DataTypeTok{title =} \StringTok{"Histogram for weight"}\NormalTok{,}\DataTypeTok{x=}\StringTok{"weight"}\NormalTok{,}\DataTypeTok{y=}\StringTok{"count"}\NormalTok{,}\KeywordTok{xlim}\NormalTok{(}\KeywordTok{c}\NormalTok{(}\DecValTok{4}\NormalTok{,}\DecValTok{7}\NormalTok{)))}\OperatorTok{+}
\KeywordTok{scale_fill_gradient}\NormalTok{(}\StringTok{"Count"}\NormalTok{,}\DataTypeTok{low=}\StringTok{"green"}\NormalTok{,}\DataTypeTok{high =} \StringTok{"red"}\NormalTok{)}
\end{Highlighting}
\end{Shaded}

\includegraphics[height=5cm]{images/ggplot_histogramv3.png}

\end{frame}

\begin{frame}[fragile]{Histograma - libreria ggplot2 qplot() vs
ggplot()}

qplot() tiene una sintaxis más sencilla con respecto a ggplot(). qplot
viene de ``quick plot''. Sin embargo, ggplot() es maś flexible y permite
configurar otros parámetros. Generalmente se usa ggplot()

\begin{Shaded}
\begin{Highlighting}[]
\KeywordTok{ggplot}\NormalTok{(}\DataTypeTok{data=}\NormalTok{PlantGrowth,}\KeywordTok{aes}\NormalTok{(PlantGrowth}\OperatorTok{$}\NormalTok{weight))}\OperatorTok{+}
\KeywordTok{geom_histogram}\NormalTok{(}\DataTypeTok{bins =} \DecValTok{7}\NormalTok{,}\DataTypeTok{col=}\StringTok{"gray"}\NormalTok{,}\DataTypeTok{fill=}\StringTok{"gold"}\NormalTok{,}\DataTypeTok{alpha=}\FloatTok{0.4}\NormalTok{,}\KeywordTok{aes}\NormalTok{(}\DataTypeTok{y=}\NormalTok{..density..))}\OperatorTok{+}
\KeywordTok{labs}\NormalTok{(}\DataTypeTok{title =} \StringTok{"Histogram for weight"}\NormalTok{,}\DataTypeTok{x=}\StringTok{"weight"}\NormalTok{,}\DataTypeTok{y=}\StringTok{"count"}\NormalTok{,}\KeywordTok{xlim}\NormalTok{(}\KeywordTok{c}\NormalTok{(}\DecValTok{4}\NormalTok{,}\DecValTok{7}\NormalTok{)))}\OperatorTok{+}
\StringTok{  }\KeywordTok{geom_density}\NormalTok{(}\DataTypeTok{col=}\DecValTok{1}\NormalTok{)}
\end{Highlighting}
\end{Shaded}

\includegraphics[height=4cm]{images/ggplot_histogram_trendline.png}

\end{frame}

\begin{frame}[fragile]{Histograma - libreria plotly}

\begin{Shaded}
\begin{Highlighting}[]
\KeywordTok{install.packages}\NormalTok{(}\StringTok{"plotly"}\NormalTok{)}
\KeywordTok{library}\NormalTok{(plotly)}
\CommentTok{#configurar usuario y key}
\CommentTok{#Sys.setenv("plotly_username"=usuario)}
\CommentTok{#Sys.setenv("plotly_api_key"=keygenerada)}
\NormalTok{p<-}\KeywordTok{plot_ly}\NormalTok{(}\DataTypeTok{x=}\NormalTok{PlantGrowth}\OperatorTok{$}\NormalTok{weight,}\DataTypeTok{type=}\StringTok{"histogram"}\NormalTok{)}
\NormalTok{char_link =}\KeywordTok{api_create}\NormalTok{(p,}\DataTypeTok{filename =} \StringTok{"histogram_basic"}\NormalTok{)}
\NormalTok{char_link}
\end{Highlighting}
\end{Shaded}

\url{https://plot.ly/~phreyes/1#plot}

\end{frame}

\begin{frame}{Diagramas de caja - Boxplot}

Es uno de los gráficos más comunes. Da un resumen bueno de una o varias
variables numéricas.\\

\begin{itemize}
\item La línea que divide la caja en la mitad representa la mediana. 
\item Los finales de la caja representan el primer $Q_{1}$ y el tercer cuartil $Q_{3}$.
\item Las líneas externas representan el valor más alto y más bajo (excluyendo valores atípicos)
\end{itemize}

\includegraphics{Agrosavia_Theme_Graficos_Curso2_files/figure-beamer/unnamed-chunk-25-1.pdf}

\end{frame}

\begin{frame}[fragile]{Diagramas de caja - Boxplot}

\begin{Shaded}
\begin{Highlighting}[]
\KeywordTok{boxplot}\NormalTok{(PlantGrowth}\OperatorTok{$}\NormalTok{weight,}\DataTypeTok{col =} \StringTok{"darkturquoise"}\NormalTok{)}
\end{Highlighting}
\end{Shaded}

\includegraphics{Agrosavia_Theme_Graficos_Curso2_files/figure-beamer/unnamed-chunk-26-1.pdf}

\end{frame}

\begin{frame}[fragile]{Diagramas de caja - Boxplot - Estructura}

\begin{Shaded}
\begin{Highlighting}[]
\KeywordTok{str}\NormalTok{(}\KeywordTok{boxplot}\NormalTok{(PlantGrowth}\OperatorTok{$}\NormalTok{weight,}\DataTypeTok{plot =} \OtherTok{FALSE}\NormalTok{))}
\end{Highlighting}
\end{Shaded}

\begin{verbatim}
## List of 6
##  $ stats: num [1:5, 1] 3.59 4.53 5.15 5.54 6.31
##  $ n    : num 30
##  $ conf : num [1:2, 1] 4.86 5.45
##  $ out  : num(0) 
##  $ group: num(0) 
##  $ names: chr "1"
\end{verbatim}

\begin{Shaded}
\begin{Highlighting}[]
\KeywordTok{quantile}\NormalTok{(PlantGrowth}\OperatorTok{$}\NormalTok{weight,}\KeywordTok{c}\NormalTok{(}\DecValTok{0}\NormalTok{,}\FloatTok{0.25}\NormalTok{,}\FloatTok{0.5}\NormalTok{,}\FloatTok{0.75}\NormalTok{,}\DecValTok{1}\NormalTok{))}
\end{Highlighting}
\end{Shaded}

\begin{verbatim}
##    0%   25%   50%   75%  100% 
## 3.590 4.550 5.155 5.530 6.310
\end{verbatim}

\end{frame}

\begin{frame}[fragile]{Diagramas de caja - Boxplot dividiendo por grupo}

\begin{Shaded}
\begin{Highlighting}[]
\KeywordTok{boxplot}\NormalTok{(weight }\OperatorTok{~}\StringTok{ }\NormalTok{group, }\DataTypeTok{data =}\NormalTok{ PlantGrowth, }\DataTypeTok{col =} \StringTok{"lightgray"}\NormalTok{)}
\end{Highlighting}
\end{Shaded}

\includegraphics{Agrosavia_Theme_Graficos_Curso2_files/figure-beamer/unnamed-chunk-28-1.pdf}

\begin{Shaded}
\begin{Highlighting}[]
\KeywordTok{boxplot}\NormalTok{(weight }\OperatorTok{~}\StringTok{ }\NormalTok{group, }\DataTypeTok{data =}\NormalTok{ PlantGrowth, }\DataTypeTok{col=}\KeywordTok{c}\NormalTok{(}\StringTok{"darkorchid"}\NormalTok{,}\StringTok{"darkturquoise"}\NormalTok{,}
                                                  \StringTok{"deepskyblue"}\NormalTok{))}
\end{Highlighting}
\end{Shaded}

\includegraphics{Agrosavia_Theme_Graficos_Curso2_files/figure-beamer/unnamed-chunk-28-2.pdf}

\end{frame}

\begin{frame}[fragile]{Diagramas de caja - Boxplot para un dataframe}

Otro conjunto de datos \textit{trees}. Medidas del diámetro, altura y
volumen de 31 arboles de mora.

\begin{Shaded}
\begin{Highlighting}[]
\KeywordTok{boxplot}\NormalTok{(trees,}\DataTypeTok{col=}\KeywordTok{c}\NormalTok{(}\StringTok{"darkorchid"}\NormalTok{,}\StringTok{"darkturquoise"}\NormalTok{,}\StringTok{"deepskyblue"}\NormalTok{))}
\end{Highlighting}
\end{Shaded}

\includegraphics{Agrosavia_Theme_Graficos_Curso2_files/figure-beamer/unnamed-chunk-29-1.pdf}
En caso de usar una matriz se realiza el boxplot por columnas.

\tiny{El diámetro esta etiquetado incorrectamente como Girth}

\end{frame}

\begin{frame}[fragile]{Diagramas de caja - Boxplot número de muestras}

\begin{Shaded}
\begin{Highlighting}[]
\NormalTok{a=}\KeywordTok{boxplot}\NormalTok{(PlantGrowth}\OperatorTok{$}\NormalTok{weight}\OperatorTok{~}\NormalTok{PlantGrowth}\OperatorTok{$}\NormalTok{group,}
          \DataTypeTok{col=}\KeywordTok{c}\NormalTok{(}\StringTok{"darkorchid"}\NormalTok{,}\StringTok{"darkturquoise"}\NormalTok{,}\StringTok{"deepskyblue"}\NormalTok{),}
          \DataTypeTok{ylim=}\KeywordTok{c}\NormalTok{(}\FloatTok{3.5}\NormalTok{,}\FloatTok{7.5}\NormalTok{))}
\KeywordTok{text}\NormalTok{(}\KeywordTok{c}\NormalTok{(}\DecValTok{1}\OperatorTok{:}\KeywordTok{nlevels}\NormalTok{(PlantGrowth}\OperatorTok{$}\NormalTok{group)),a}\OperatorTok{$}\NormalTok{stats[}\KeywordTok{nrow}\NormalTok{(a}\OperatorTok{$}\NormalTok{stats),]}\OperatorTok{+}\FloatTok{0.5}\NormalTok{,}
     \KeywordTok{paste}\NormalTok{(}\StringTok{"n="}\NormalTok{,a}\OperatorTok{$}\NormalTok{n[}\DecValTok{1}\OperatorTok{:}\KeywordTok{nlevels}\NormalTok{(PlantGrowth}\OperatorTok{$}\NormalTok{group)],}\DataTypeTok{sep=}\StringTok{""}\NormalTok{))}
\end{Highlighting}
\end{Shaded}

\includegraphics{Agrosavia_Theme_Graficos_Curso2_files/figure-beamer/unnamed-chunk-30-1.pdf}

\end{frame}

\begin{frame}[fragile]{Diagramas de caja - Horizontal}

\begin{Shaded}
\begin{Highlighting}[]
\KeywordTok{boxplot}\NormalTok{(trees,}\DataTypeTok{horizontal=}\OtherTok{TRUE}\NormalTok{,}\DataTypeTok{col=}\KeywordTok{c}\NormalTok{(}\StringTok{"darkorchid"}\NormalTok{,}\StringTok{"darkturquoise"}
\NormalTok{                                    ,}\StringTok{"deepskyblue"}\NormalTok{))}
\end{Highlighting}
\end{Shaded}

\includegraphics{Agrosavia_Theme_Graficos_Curso2_files/figure-beamer/unnamed-chunk-31-1.pdf}

\end{frame}

\begin{frame}[fragile]{Diagramas de caja - Horizontal}

\begin{Shaded}
\begin{Highlighting}[]
\KeywordTok{boxplot}\NormalTok{(trees,}\DataTypeTok{horizontal=}\OtherTok{TRUE}\NormalTok{,}\DataTypeTok{col=}\KeywordTok{c}\NormalTok{(}\StringTok{"darkorchid"}\NormalTok{,}\StringTok{"darkturquoise"}
\NormalTok{                             ,}\StringTok{"deepskyblue"}\NormalTok{),}\DataTypeTok{outline =} \OtherTok{FALSE}\NormalTok{)}
\end{Highlighting}
\end{Shaded}

\includegraphics{Agrosavia_Theme_Graficos_Curso2_files/figure-beamer/unnamed-chunk-32-1.pdf}

\end{frame}

\begin{frame}[fragile]{Diagramas de caja - 2 variables dividiendo por
grupo}

\begin{Shaded}
\begin{Highlighting}[]
\CommentTok{# Creemos columna dosis para ver otro ejemplo tratamientos y dosis }
\CommentTok{# Números aleatorios para la dosis por esto la gráfica va a cambiar }
\CommentTok{# para cada uno}
\NormalTok{PlantGrowth}\OperatorTok{$}\NormalTok{dose=}\KeywordTok{sample}\NormalTok{(}\DecValTok{1}\OperatorTok{:}\DecValTok{3}\NormalTok{,}\DecValTok{30}\NormalTok{,}\DataTypeTok{replace=}\OtherTok{TRUE}\NormalTok{)}
\CommentTok{# Boxplot dividiendo por grupo}
\KeywordTok{boxplot}\NormalTok{(weight }\OperatorTok{~}\StringTok{ }\NormalTok{dose}\OperatorTok{:}\NormalTok{group, }\DataTypeTok{data =}\NormalTok{ PlantGrowth,}
\DataTypeTok{boxwex =} \FloatTok{0.5}\NormalTok{, }\DataTypeTok{col =} \KeywordTok{c}\NormalTok{(}\StringTok{"red"}\NormalTok{, }\StringTok{"green"}\NormalTok{,}\StringTok{"blue"}\NormalTok{),}
\DataTypeTok{main =} \StringTok{"Plant Growth"}\NormalTok{,}
\DataTypeTok{xlab =} \StringTok{" dose (mg)"}\NormalTok{, }\DataTypeTok{ylab =} \StringTok{"weight (kg)"}\NormalTok{,}
\DataTypeTok{sep =} \StringTok{":"}\NormalTok{, }\DataTypeTok{lex.order =} \OtherTok{TRUE}\NormalTok{, }\DataTypeTok{ylim =} \KeywordTok{c}\NormalTok{(}\DecValTok{0}\NormalTok{, }\DecValTok{8}\NormalTok{), }\DataTypeTok{yaxs =} \StringTok{"i"}\NormalTok{)}
\end{Highlighting}
\end{Shaded}

\end{frame}

\begin{frame}{Diagramas de caja - agregar}

\includegraphics{Agrosavia_Theme_Graficos_Curso2_files/figure-beamer/unnamed-chunk-34-1.pdf}

\end{frame}

\begin{frame}[fragile]{Diagramas de caja - libreria lattice}

\begin{Shaded}
\begin{Highlighting}[]
\KeywordTok{bwplot}\NormalTok{(}\OperatorTok{~}\NormalTok{weight, }\DataTypeTok{data=}\NormalTok{PlantGrowth,}\DataTypeTok{xlab=}\StringTok{"weight (kg)"}\NormalTok{,}
\DataTypeTok{main=}\StringTok{"Plant Growth by treatment"}\NormalTok{)}
\end{Highlighting}
\end{Shaded}

\includegraphics[height=5cm]{images/boxplot_lattice.png}

\end{frame}

\begin{frame}[fragile]{Diagramas de caja - libreria lattice}

\begin{Shaded}
\begin{Highlighting}[]
\KeywordTok{bwplot}\NormalTok{(}\OperatorTok{~}\NormalTok{weight}\OperatorTok{|}\StringTok{ }\NormalTok{group, }\DataTypeTok{data=}\NormalTok{PlantGrowth,}\DataTypeTok{xlab=}\StringTok{"weight (kg)"}\NormalTok{,}
\DataTypeTok{main=}\StringTok{"Plant Growth by treatment"}\NormalTok{,}\DataTypeTok{layout=}\KeywordTok{c}\NormalTok{(}\DecValTok{3}\NormalTok{,}\DecValTok{1}\NormalTok{))}
\end{Highlighting}
\end{Shaded}

\includegraphics[height=5cm]{images/boxplot_lattice_3.png}

\end{frame}

\begin{frame}[fragile]{Diagramas de caja - libreria ggplot2}

\begin{Shaded}
\begin{Highlighting}[]
\KeywordTok{ggplot}\NormalTok{(PlantGrowth,}\KeywordTok{aes}\NormalTok{(}\DataTypeTok{x=}\NormalTok{group,}\DataTypeTok{y=}\NormalTok{weight,}\DataTypeTok{fill=}\NormalTok{group))}\OperatorTok{+}\KeywordTok{geom_boxplot}\NormalTok{(}\DataTypeTok{alpha=}\FloatTok{0.3}\NormalTok{)}
\end{Highlighting}
\end{Shaded}

\includegraphics[height=5cm]{images/boxplot_Basicoggplot.png}

\end{frame}

\begin{frame}[fragile]{Histograma y Diagramas de caja - libreria
ggplot2}

\begin{Shaded}
\begin{Highlighting}[]
\CommentTok{#Dividir la pantalla}
\KeywordTok{layout}\NormalTok{(}\DataTypeTok{mat=}\KeywordTok{matrix}\NormalTok{(}\KeywordTok{c}\NormalTok{(}\DecValTok{1}\NormalTok{,}\DecValTok{2}\NormalTok{),}\DecValTok{2}\NormalTok{,}\DecValTok{1}\NormalTok{,}\DataTypeTok{byrow=}\OtherTok{TRUE}\NormalTok{),}\DataTypeTok{height=}\KeywordTok{c}\NormalTok{(}\DecValTok{1}\NormalTok{,}\DecValTok{8}\NormalTok{))}
\CommentTok{#Dibujar el boxplot y el histograma}
\KeywordTok{par}\NormalTok{(}\DataTypeTok{mar=}\KeywordTok{c}\NormalTok{(}\DecValTok{0}\NormalTok{,}\FloatTok{3.1}\NormalTok{,}\FloatTok{1.1}\NormalTok{,}\FloatTok{2.1}\NormalTok{))}
\KeywordTok{boxplot}\NormalTok{(PlantGrowth}\OperatorTok{$}\NormalTok{weight,}\DataTypeTok{horizontal =} \OtherTok{TRUE}\NormalTok{,}\DataTypeTok{xaxt=}\StringTok{"n"}\NormalTok{,}
        \DataTypeTok{col=}\NormalTok{(}\StringTok{"mediumvioletred"}\NormalTok{),}\DataTypeTok{frame=}\NormalTok{F)}
\KeywordTok{par}\NormalTok{(}\DataTypeTok{mar=}\KeywordTok{c}\NormalTok{(}\DecValTok{4}\NormalTok{,}\FloatTok{3.1}\NormalTok{,}\FloatTok{1.1}\NormalTok{,}\FloatTok{2.1}\NormalTok{))}
\KeywordTok{hist}\NormalTok{(PlantGrowth}\OperatorTok{$}\NormalTok{weight,}\DataTypeTok{col=}\StringTok{"mediumturquoise"}\NormalTok{,}\DataTypeTok{border=}\NormalTok{F,}\DataTypeTok{main=}\StringTok{""}\NormalTok{,}\DataTypeTok{xlab=}\StringTok{"weight"}\NormalTok{)}
\end{Highlighting}
\end{Shaded}

\includegraphics[height=5cm]{images/HistogramaBoxplotGGplot.png}

\end{frame}

\begin{frame}[fragile]{Diagramas de caja - plotly}

\begin{Shaded}
\begin{Highlighting}[]
\NormalTok{p<-}\KeywordTok{plot_ly}\NormalTok{(PlantGrowth,}\DataTypeTok{y=}\OperatorTok{~}\NormalTok{weight,}\DataTypeTok{color=}\OperatorTok{~}\NormalTok{group,}\DataTypeTok{type=}\StringTok{"box"}\NormalTok{)}
\NormalTok{chart_link =}\StringTok{ }\KeywordTok{api_create}\NormalTok{(p, }\DataTypeTok{filename=}\StringTok{"box-multiple"}\NormalTok{)}
\NormalTok{chart_link}
\end{Highlighting}
\end{Shaded}

\url{https://plot.ly/~phreyes/3/}

\end{frame}

\section{Ejercicios prácticos}\label{ejercicios-practicos}

\begin{frame}{Ejercicios prácticos - Diagramas de distribución 1}

\begin{itemize}
\item Inicialmente trabajaremos con un conjunto de datos predefinido en R.
\item Si queda tiempo la idea es trabajar con un conjunto de datos de cada uno.
\end{itemize}

\end{frame}

\begin{frame}{Ejercicios prácticos - Diagramas de distribución}

Conjunto de datos \textit{ToothGrowth}. Este conjunto de datos consiste
en la longitud de los dientes de cuyes. Se dividieron en grupos al los
que se les dieron diferentes dosis de vitamina C (0.5, 1, y 2 mg)
mediante dos métodos diferentes (jugo de naranja OJ o ácido ascórbico
VC).

\begin{itemize}
\item Probar con este nuevo conjunto de datos a realizar las gráficas que se encuentran en las \href{}{diapositivas}. De histogramas y diagramas de caja.
\end{itemize}

\end{frame}

\section{Referencias}\label{referencias}

\begin{frame}{Referencias}

\begin{itemize}
\item help r
\item \underline{\textcolor{blue}{\href{https://www.r-graph-gallery.com/}{The R Graph Gallery}}}
\end{itemize}

\includegraphics[height=4cm]{images/RGraphGallery.png}

\end{frame}

\end{document}
